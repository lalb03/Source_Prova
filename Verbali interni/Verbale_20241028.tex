\documentclass{article}
\usepackage[utf8]{inputenc}
\usepackage[absolute]{textpos}
\usepackage[default]{raleway}
\usepackage{booktabs}
\usepackage{array}
\usepackage{titlesec, comment, tabularx, makecell, listings, array, setspace, geometry, graphicx, xcolor, xparse, fancyvrb, relsize, fancyhdr, booktabs, hyperref}
\usepackage{colortbl}
\geometry{a4paper, left=2cm, right=2cm, top=1cm, bottom=1.8cm}
\renewcommand{\headrulewidth}{0pt}

% Definisci uno stile per i comandi git
\definecolor{light-gray}{gray}{0.92}

\lstdefinestyle{code}{
    frame=single,
    framesep=1mm,
    rulecolor=\color{light-gray},
    backgroundcolor=\color{light-gray},
    basicstyle=\ttfamily,
}

% ----------------------------- Definizione tabella ---------------------------

\newcolumntype{C}[1]{>{\centering\arraybackslash}m{#1}}

%\setcellgapes{2ex} % Imposta l'altezza dell'header (2ex)


% ------------------------------Metadati indice --------------------------------
\title{\textbf{\fontsize{28}{6}\selectfont Indice}}
\author{\fontsize{14}{6}\selectfont Byte Your Dreams}
\date{ottobre 21, 2024}

% -----------------------------Creazione footer --------------------------------

\pagestyle{fancy}
\fancyhf{}
\renewcommand{\footrulewidth}{0.4pt}
\lfoot{
    \parbox[c]{1.5cm}{\includegraphics[width=1.5cm]{./../images/byd_logo.png}}
}
\rfoot{\thepage}


% -----------------------------Creazione footer --------------------------------

%\pagestyle{fancy}
%\fancyhf{}
%\renewcommand{\footrulewidth}{0.4pt}
%\cfoot{\thepage}

% --------------------------Modifica formato hyperlinks ------------------------

\hypersetup{
    colorlinks=true,
    linkcolor=black,
    filecolor=black,      
    pdftitle={Verbale Interno 21/10/2024},  %inserisci data verbale
    pdfpagemode=FullScreen,
}

% ------------------------------- Valore sotto-paragrafi indice --------------------------------------

\setcounter{secnumdepth}{4}
\setcounter{tocdepth}{4}

\titleformat{\section}
{\normalfont\huge\bfseries}{\thesection}{0.2cm}{}
\titlespacing*{\paragraph}{0pt}{0.5cm}{0.1cm}

\titleformat{\subsection}
{\normalfont\Large\bfseries}{\thesubsection}{0.2cm}{}
\titlespacing*{\paragraph}{0pt}{0.5cm}{0.1cm}

\titleformat{\subsubsection}
{\normalfont\large\bfseries}{\thesubsubsection}{0.2cm}{}
\titlespacing*{\paragraph}{0pt}{0.5cm}{0.1cm}

\titleformat{\paragraph}
{\normalfont\normalsize\bfseries}{\theparagraph}{0.2cm}{}
\titlespacing*{\paragraph}{0pt}{0.5cm}{0.1cm}

% ------------------------------- Front Page ---------------------------------------


\begin{document}

% --------------------------Aggiunta firma finale ------------------------
\begin{textblock*}{\textwidth}(0.78\textwidth, 1\textheight)
    Il responsabile: L. Albertin\\
    
\end{textblock*}
% ------------------------------------------------------------------------
\pagestyle{fancy}
\begin{center}
\includegraphics[width = 0.7\textwidth]{./../images/BYD_logo-logotipo.png} \\
\fontsize{12}{6}\textcolor[RGB]{60, 60, 60}{\textit{byteyourdreams.swe@gmail.com}} \\
\vspace{0.5cm}
\fontsize{16}{6}\selectfont Verbale Interno $\cdot$ Data: 21/10/2024 \\
\vspace{0.5cm}
\end{center}




\section*{Informazioni documento}
\def\arraystretch{1.2}
\begin{tabular}{>{\raggedleft\arraybackslash}p{0.2\textwidth}|>{\raggedright\arraybackslash}p{0.6\textwidth}c}
\hline
\addlinespace
\textbf{Luogo} & Discord \vspace{10pt} \\
\textbf{Orario} & 18:00 - 20:00 \vspace{10pt} \\
\textbf{Redattore} & A.M. Margarit \\ & L. Zanesco \vspace{10pt} \\
\textbf{Verificatore} & A. Mio \\ & O.F. Stiglet \\ & Y. Huang \vspace{10pt} \\
\textbf{Amministratore} & L. Albertin \vspace{10pt} \\
\textbf{Destinatari} & T. Vardanega \\ & R. Cardin \vspace{10pt} \\
\textbf{Partecipanti} & L. Albertin \\ & A. Mio \\ & O.F. Stiglet \\ & L. Zanesco \\ & Y. Huang\\ & A.M. Margarit \\\vspace{10pt} \\
\end{tabular}

\pagebreak 
% ------------------------- Changelog ----------------------------

\section*{Registro delle modifiche}
\begin{center}
    \begin{tabular}{|C{2.5cm}|C{2.5cm}|C{2.5cm}|C{2.5cm}|C{2.5cm}|}
    \hline
    \textbf{Versione} & \textbf{Data} & \textbf{Autore} & \textbf{Verificatore} & \textbf{Dettaglio} \\ \hline
    0.0.2 & 25/10/2024 & A.M. Margarit & Y. Huang & Riscrittura in latex\\ \hline
    0.0.1 & 16/10/2024 & \makecell{L. Zanesco \\ A.M. Margarit} & O.F. Stiglet & Prima redazione\\
    \hline
\end{tabular}
\end{center}

\pagebreak

% ------------------------- Generazione automatica indice ----------------------
\setstretch{1.5}
\maketitle
\thispagestyle{fancy}
\tableofcontents
\setstretch{1.2}
\pagebreak

% ------------------------ INIZIO DOCUMENTO ----------------------
\flushleft

\section{Revisione del periodo precedente}
Revisionando \textit{attività}\textsubscript{G} svolte fino ad ora, si è notato che la metodologia per la redazione dei documenti attuata fino ad ora non era ottimale. La trasposizione da Docs a \latex richiede una rielaborazione del testo, aggiungendo un ulteriore rischio di errori e richiedendo un'ulteriore verifica.

\section{Ordine del giorno}
    \subsection{Revisione documenti prodotti}
    Tutti i documenti redatti fino al momento dell’incontro, sono stati verificati e saranno caricati nella \textit{repository}\textsubscript{}{G} nei giorni seguenti.
    \subsection{Rivalutazione metodologie per la redazione dei documenti}
    Si è deciso di creare direttamente i documenti in \latex. Questi non saranno caricati più su Google Drive, ma su una \textit{repository}\textsubscript{G} privata in modo da rendere più efficiente la modifica dei file. Una volta validati, verranno generati i file PDF e caricati nella \textit{repository}\textsubscript{G} principale.
    \subsection{Redazione documento “Analisi dei Capitolati”}
    E’ stato redatto per intero il documento “\textbf{Analisi dei Capitolati}”. Durante la redazione, sono state fatte nuove considerazioni riguardo alcuni capitolati precedentemente valutati.
    \subsectino{Prima stesura documenti “Preventivo Costi e assunzione degli impegni” e “Lettera di Presentazione Candidatura”}
    Dopo una discussione sull'assunzione degli impegni per ogni singolo componente del gruppo, si è iniziata la stesura del relativo documento. E' stato inoltre redatto il documento "\textbf{Lettera di Presentazione}". 

\section{Attività da svolgere}
Non sono state definite nuove \textit{attività}\textsubscript{G} da svolgere.

\end{document}