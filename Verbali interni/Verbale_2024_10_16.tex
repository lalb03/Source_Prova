\documentclass{article}
\usepackage[utf8]{inputenc}
\usepackage[absolute]{textpos}
\usepackage[default]{raleway}
\usepackage{booktabs}
\usepackage{array}
\usepackage{titlesec, comment, tabularx, makecell, listings, array, setspace, geometry, graphicx, xcolor, xparse, fancyvrb, relsize, fancyhdr, booktabs, hyperref}
\usepackage{colortbl}
\geometry{a4paper, left=2cm, right=2cm, top=1cm, bottom=1.8cm}
\renewcommand{\headrulewidth}{0pt}

% Definisci uno stile per i comandi git
\definecolor{light-gray}{gray}{0.92}

\lstdefinestyle{code}{
    frame=single,
    framesep=1mm,
    rulecolor=\color{light-gray},
    backgroundcolor=\color{light-gray},
    basicstyle=\ttfamily,
}

% ----------------------------- Definizione tabella ---------------------------

\newcolumntype{C}[1]{>{\centering\arraybackslash}m{#1}}

%\setcellgapes{2ex} % Imposta l'altezza dell'header (2ex)


% ------------------------------Metadati indice --------------------------------
\title{\textbf{\fontsize{28}{6}\selectfont Indice}}
\author{\fontsize{14}{6}\selectfont Byte Your Dreams}
\date{ottobre 16, 2024}

% -----------------------------Creazione footer --------------------------------

\pagestyle{fancy}
\fancyhf{}
\renewcommand{\footrulewidth}{0.4pt}
\lfoot{
    \parbox[c]{1.5cm}{\includegraphics[width=1.5cm]{./../images/byd_logo.png}}
}
\rfoot{\thepage}


% -----------------------------Creazione footer --------------------------------

%\pagestyle{fancy}
%\fancyhf{}
%\renewcommand{\footrulewidth}{0.4pt}
%\cfoot{\thepage}

% --------------------------Modifica formato hyperlinks ------------------------

\hypersetup{
    colorlinks=true,
    linkcolor=black,
    filecolor=black,      
    pdftitle={Verbale Interno 16/10/2024},  %inserisci data verbale
    pdfpagemode=FullScreen,
}

% ------------------------------- Valore sotto-paragrafi indice --------------------------------------

\setcounter{secnumdepth}{4}
\setcounter{tocdepth}{4}

\titleformat{\section}
{\normalfont\huge\bfseries}{\thesection}{0.2cm}{}
\titlespacing*{\paragraph}{0pt}{0.5cm}{0.1cm}

\titleformat{\subsection}
{\normalfont\Large\bfseries}{\thesubsection}{0.2cm}{}
\titlespacing*{\paragraph}{0pt}{0.5cm}{0.1cm}

\titleformat{\subsubsection}
{\normalfont\large\bfseries}{\thesubsubsection}{0.2cm}{}
\titlespacing*{\paragraph}{0pt}{0.5cm}{0.1cm}

\titleformat{\paragraph}
{\normalfont\normalsize\bfseries}{\theparagraph}{0.2cm}{}
\titlespacing*{\paragraph}{0pt}{0.5cm}{0.1cm}

% ------------------------------- Front Page ---------------------------------------


\begin{document}

% --------------------------Aggiunta firma finale ------------------------
\begin{textblock*}{\textwidth}(0.78\textwidth, 1\textheight)
    Il responsabile: L. Albertin
\end{textblock*}
% ------------------------------------------------------------------------
\pagestyle{fancy}
\begin{center}
\includegraphics[width = 0.7\textwidth]{./../images/BYD_logo-logotipo.png} \\
\fontsize{12}{6}\textcolor[RGB]{60, 60, 60}{\textit{byteyourdreams.swe@gmail.com}} \\
\vspace{0.5cm}
\fontsize{16}{6}\selectfont Verbale Interno $\cdot$ Data: 16/10/2024 \\
\vspace{0.5cm}
\end{center}




\section*{Informazioni documento}
\def\arraystretch{1.2}
\begin{tabular}{>{\raggedleft\arraybackslash}p{0.2\textwidth}|>{\raggedright\arraybackslash}p{0.6\textwidth}c}
\hline
\addlinespace
\textbf{Luogo} & Discord \vspace{10pt} \\
\textbf{Orario} & 17:00 - 20:00 \vspace{10pt} \\
\textbf{Redattore} & A.M. Margarit\\ & L. Zanesco \vspace{10pt} \\
\textbf{Verificatore} & A. Mio \\ & O.F. Stiglet \\ & Y. Huang \vspace{10pt} \\
\textbf{Amministratore} & L. Albertin \vspace{10pt} \\
\textbf{Destinatari} & T. Vardanega \\ & R. Cardin \vspace{10pt} \\
\textbf{Partecipanti} & L. Albertin \\ & A. Mio \\ & O.F. Stiglet \\ & L. Zanesco \\ & Y. Huang\\ & A.M. Margarit \\\vspace{10pt} \\
\end{tabular}

\pagebreak 
% ------------------------- Changelog ----------------------------

\section*{Registro delle modifiche}
\begin{center}
    \begin{tabular}{|C{2.5cm}|C{2.5cm}|C{2.5cm}|C{2.5cm}|C{2.5cm}|}
    \hline
    \textbf{Versione} & \textbf{Data} & \textbf{Autore} & \textbf{Verificatore} & \textbf{Dettaglio} \\ \hline
    0.0.2 & 25/10/2024 & L. Zanesco & A. Mio & Riscrittura in latex\\ \hline
    0.0.1 & 16/10/2024 & \makecell{L. Zanesco \\ A.M. Margarit} & L. Albertin & Prima redazione\\
    \hline
\end{tabular}
\end{center}

\pagebreak

% ------------------------- Generazione automatica indice ----------------------
\setstretch{1.5}
\maketitle
\thispagestyle{fancy}
\tableofcontents
\setstretch{1.2}
\pagebreak

% ------------------------ INIZIO DOCUMENTO ----------------------
\flushleft

\section{Revisione del periodo precedente}
In considerazione del fatto che questo costituisce il nostro primo incontro e che nessuna \textit{attività}\textsubscript{G} è stata ancora avviata, ci è preclusa la possibilità di condurre una revisione di quanto fatto nel periodo precedente.

\section{Ordine del giorno}
    \subsection{Scelta nome e logo}
    Dopo che ogni componente si è presentato al resto del gruppo, si è discusso sul nome e sul logo di quest’ultimo.
    \subsection{Analisi e discussione capitolati proposti (più quotati)}
    Dopo che ogni componente del gruppo ha espresso le proprie preferenze sono stati identificati i capitolati più votati:
    \textbf{\begin{enumerate}
        \item C2 - Vimar GENIALE
        \item C5 - 3Dataviz
        \item C9 - BuddyBot
    \end{enumerate}}
    \subsection{Definizione canali di comunicazione}
    Si è scelto di utilizzare \textit{Discord} e \textit{Whatsapp} come canali di comunicazione.
    
    \subsection{Programmazione riunioni}
    Si è deciso di fissare delle riunioni settimanali, salvo necessità, per discutere di quanto è stato affrontato durante la settimana, e per decidere le successive \textit{attività}\textsubscript{G} da svolgere.\\
    Il prossimo incontro si terrà il 21/10/2024 su \textit{Discord}.
    
    \subsection{Definizione linee guida}
    Ogni incontro è guidato da domande riguradanti il \textit{Way Of Working}\textsubscript{G} del gruppo e dei singoli:
    \textbf{\begin{itemize}
        \item Come sono state svolte le attività assegnate?
        \item C’è qualcosa che ne ha impedito lo svolgimento?
        \item È possibile attuare miglioramenti?
        \item Quali sono le successive attività da svolgere?
        \item Chi le dovrà svolgere?
        \item Come dovranno essere svolte?
    \end{itemize}}
    \pagebreak
    \subsection{Definizione ruoli componente}
    Fino alla data di aggiudicazione degli appalti, i ruoli di ciascun componente, per la stesura dei documenti, non cambieranno. Ruolo e nome di ogni componente sono visibili nella prima pagina.\\
    I ruoli che si sono distinti sono riportati a seguito, assieme ad una breve descrizione:
    \begin{itemize}
        \item[$-$] \textbf{Amministratore}: definisce, controlla, e mantiene un corretto \textit{Way Of Working}\textsubscript{G}. In questo periodo svolge anche il ruolo del responsabile, ossia definisce i to-do con le relative scadenze  e priorità. Inoltre rappresenta il gruppo.
        \item[$-$] \textbf{Redattore}: scrive i documenti.
        \item[$-$] \textbf{Verificatore}: valida i documenti e li rilascia nella \textit{Repository}\textsubscript{G}
    \end{itemize}
    \subsection{Metodologia produzione e gestione documenti}
    Per la realizzazione dei documenti, si è concordato, tra i membri, un modello prestabilito per facilitare i redattori, e per garantire l’omogeneità di quanto scritto nel documenti.\\
    Ogni documento prima di essere rilasciato nella \textit{Repository}\textsubscript{G}, è gestito in una cartella condivisa su \textit{Google Drive}. Dopo che un documento è completato, viene validato dai verificatori, redatto con \LaTeX e infine caricato nella \textit{Repository}\textsubscript{G}.
    \subsection{Programmazione incontri con le aziende}
    Sono state documentate in un file le domande riguardati dubbi e/o incertezze sui capitolati, non solo per quelli più votati. Sono state inoltre definite delle bozze email da inviare alle aziende il giorno 21/10/2024.
\section{Attività da svolgere}
    \begin{center}
        \begin{tabular}{|C{7cm}|C{1,5cm}|C{3cm}|}
            \hline
            \textbf{Titolo} & \textbf{N. Issue} & \textbf{Verificatore} \\
            \hline
            Valutazione Capitolati & 1 & A. Mio \\ \hline
            Preventivo Costi & 2 & L. Albertin \\ \hline
            Lettera di presentazione & 3 & \makecell{Y. Huang \\ O.F. Stiglet}  \\
            \hline
        \end{tabular}
    \end{center}

\end{document}