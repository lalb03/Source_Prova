\documentclass{article}
\usepackage[utf8]{inputenc}
\usepackage[absolute]{textpos}
\usepackage[default]{raleway}
\usepackage{booktabs}
\usepackage{array}
\usepackage{titlesec, comment, tabularx, makecell, listings, array, setspace, geometry, graphicx, xcolor, xparse, fancyvrb, relsize, fancyhdr, booktabs, hyperref}
\usepackage{colortbl}
\usepackage{makecell}
\usepackage{eurosym}
\geometry{a4paper, left=2cm, right=2cm, top=1cm, bottom=1.8cm}
\renewcommand{\headrulewidth}{0pt}

% Definisci uno stile per i comandi git
\definecolor{light-gray}{gray}{0.92}

\lstdefinestyle{code}{
    frame=single,
    framesep=1mm,
    rulecolor=\color{light-gray},
    backgroundcolor=\color{light-gray},
    basicstyle=\ttfamily,
}

% ----------------------------- Definizione tabella ---------------------------

\newcolumntype{C}[1]{>{\centering\arraybackslash}m{#1}}

%\setcellgapes{2ex} % Imposta l'altezza dell'header (2ex)


% ------------------------------Metadati indice --------------------------------
\title{\textbf{\fontsize{28}{6}\selectfont Indice}}
\author{\fontsize{14}{6}\selectfont Byte Your Dreams}
\date{ottobre 29, 2024}


% -----------------------------Creazione footer --------------------------------

\pagestyle{fancy}
\fancyhf{}
\renewcommand{\footrulewidth}{0.4pt}
\lfoot{
    \parbox[c]{1.5cm}{\includegraphics[width=1.5cm]{docsSouce/images/byd_logo}}
}
\rfoot{\thepage}


% -----------------------------Creazione footer --------------------------------

%\pagestyle{fancy}
%\fancyhf{}
%\renewcommand{\footrulewidth}{0.4pt}
%\cfoot{\thepage}

% --------------------------Modifica formato hyperlinks ------------------------

\hypersetup{
    colorlinks=true,
    linkcolor=black,
    filecolor=black,      
    pdftitle={Preventivo costi e assunzione impegni},
    pdfpagemode=FullScreen,
}

% ------------------------------- Valore sotto-paragrafi indice --------------------------------------

\setcounter{secnumdepth}{4}
\setcounter{tocdepth}{4}

\titleformat{\section}
{\normalfont\huge\bfseries}{\thesection}{0.2cm}{}
\titlespacing*{\paragraph}{0pt}{0.5cm}{0.1cm}

\titleformat{\subsection}
{\normalfont\Large\bfseries}{\thesubsection}{0.2cm}{}
\titlespacing*{\paragraph}{0pt}{0.5cm}{0.1cm}

\titleformat{\subsubsection}
{\normalfont\large\bfseries}{\thesubsubsection}{0.2cm}{}
\titlespacing*{\paragraph}{0pt}{0.5cm}{0.1cm}

\titleformat{\paragraph}
{\normalfont\normalsize\bfseries}{\theparagraph}{0.2cm}{}
\titlespacing*{\paragraph}{0pt}{0.5cm}{0.1cm}

% ------------------------------- Front Page ---------------------------------------


\begin{document}

% --------------------------Aggiunta firma finale ------------------------
\begin{textblock*}{\textwidth}(0.78\textwidth, 1\textheight)
    Il responsabile: L. Albertin Ciao!
\end{textblock*}
% ------------------------------------------------------------------------
\pagestyle{fancy}
\begin{center}
\vspace*{-2cm}
\includegraphics[width = 0.7\textwidth]{docsSouce/images/byd_logo-logotipo.png}
\fontsize{12}{6}\textcolor[RGB]{60, 60, 60}{\textit{byteyourdreams.swe@gmail.com}} \\
\vspace{0.5cm}
\fontsize{16}{6}\selectfont Preventivo costi e assunzione impegni \\
\vspace{0.5cm}
\end{center}

\section*{Informazioni documento}
\def\arraystretch{1.2}
\begin{tabular}{>{\raggedleft\arraybackslash}p{0.2\textwidth}|>{\raggedright\arraybackslash}p{0.6\textwidth}c}
\hline
\addlinespace
\textbf{Data} & 29/10/2024 \vspace{10pt} \\
\textbf{Redattore} & A. Mio \vspace{10pt} \\
\textbf{Verificatore} & L. ALbertin \vspace{10pt} \\
\textbf{Amministratore} & L. Albertin \vspace{10pt} \\
\textbf{Destinatari} & T. Vardanega \\ & R. Cardin \vspace{10pt}
\end{tabular}

\pagebreak 

% ------------------------- Changelog ----------------------------


% ------------------------- Generazione automatica indice ----------------------


% ------------------------ INIZIO DOCUMENTO ----------------------
\flushleft

\section{Impegni orari}

Ogni componente del gruppo \textit{\textbf{Byte Your Dreams}} si impegna a lavorare al progetto per un monte ore totale di 92 ore produttive ciascuno.
Ogni membro proverà ogni ruolo per lo stesso ammontare di ore, in modo da suddividere equamente il carico.\\
Sottostante si trova una tabella dove sono riportati i costi orari per ruolo.

\begin{table}[h!]
\centering
\begin{tabular}{|c|c|c|c|c|}
\hline
\textbf{Ruoli} & \textbf{\makecell{Costo\\ orario\\ (\euro)}} & \textbf{\makecell{Ore\\ previste per\\ ruolo}} & \textbf{\makecell{Ore\\ previste per\\ membro}} & \textbf{\makecell{Costo per\\ ruolo\\ (\euro)}} \\ \hline
\textbf{\textit{Amministratore}}    & 30 & 48  & 8  & 1440  \\ \hline
\textbf{\textit{Responsabile}}  & 20 & 48  & 8  & 960   \\ \hline
\textbf{\textit{Analista}}        & 25 & 72  & 12  & 1800  \\ \hline
\textbf{\textit{Progettista}}     & 25 & 120 & 20 & 3000  \\ \hline
\textbf{\textit{Programmatore}}   & 15 & 114 & 19 & 1710  \\ \hline
\textbf{\textit{Verificatore}}    & 15 & 150 & 25 & 2250  \\ \hline
\textbf{TOTALE} & -  & 540 & 92 & 11160 \\ \hline
\end{tabular}
\caption{Tabella dei ruoli, costi orari e ore di lavoro}
\label{tab:ruoli_costi_dettagliati}
\end{table}
\section{Descrizione Ruoli}
\begin{itemize}
    \item  \textbf{Amministratore}: Gestisce il progetto, le risorse e le attività. Assicura l'efficienza di procedure, strumenti e tecnologie a supporto del \textit{Way of Working}\textsubscript{G}, contribuendo al miglioramento continuo del flusso di lavoro.
    \item  \textbf{Responsabile}: Supervisiona il progetto, gestendo il team e le risorse, pianificando le \textit{attività}\textsubscript{}{g}, coordinando il lavoro e assicurando il rispetto delle scadenze. Inoltre, approva il rilascio di prodotti parziali o finali e fa da collegamento tra il team e gli stakeholder;
    \item  \textbf{Analista}: Studia e definisce i requisiti del \textit{software}\textsubscript{G} interagendo con i clienti e gli stakeholder. Traducono i requisiti aziendali in specifiche tecniche chiare per gli sviluppatori;
    \item  \textbf{Progettista}: Si occupa della progettazione del \textit{sistema}\textsubscript{G}, defininendo l'\textit{architettura}\textsubscript{G}, le componenti e le loro interazioni. Si basano sulle specifiche fornite dagli analisti;
    \item  \textbf{Programmatore}: È responsabile della scrittura del codice del \textit{software}\textsubscript{G}, seguendo le specifiche fornite dai progettisti;
    \item  \textbf{Verificatore}: Esegue il controllo qualità del software\textsubscript{G}, verificando che il codice e la documentazione rispettino gli \textit{standard}\textsubscript{G} di programmazione e i requisiti specificati. Inoltre controlla che non siano presenti errori.
\end{itemize}

\section{Preventivo dei costi}
Il costo stimato per il progetto è di \textbf{11160.00\euro}. Sono state considerate la pianificazione presentata e la fase preliminare di analisi.
La stima tiene conto di ogni fase del progetto,

\section{Scadenza di consegna}
Il gruppo stima di consegnare il prodotto finito relativo al capitolato “\textbf{Vimar GENIALE}" proposto dall’azienda \textit{\textbf{Vimar}} entro e non oltre il 18/04/2025.


\end{document}
